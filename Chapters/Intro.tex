\chapter{Introduction} % Main chapter title

\label{Chapter1} % For referencing the chapter elsewhere, use \ref{Chapter1} 

The Standard Model (SM) of particle physics has proven to be a successful model to describe the behaviour and the interactions between elementary particles. Nevertheless, one had to wait long for the experimental verification of a theory predicting the masses of elementary particles (especially of the weak gauge bosons $W^\pm$ and $Z$) -- the Higgs mechanism.\\
The reason for the introduction of this mechanism was the result of an attempt to keep the gauge invariance of the SM, since an addition of a mass term to the wave equations of the force carriers results in a supplementary non-gauge invariant term. For this reason, in the early 60s Peter Higgs \parencite{Reference1}, François Englert and Robert Brout \parencite{Reference2} have introduced a mechanism (which has since become known as the Brout-Englert-Higgs mechanism), based on the assumption of a gauge-invariant potential with ground states which are not invariant under such symmetry transformations. Therfore, the symmetry of a such system is spontaneously broken, leading to different mass terms. Since these terms are proportional to the mass of the particle they couple to, it follows that with a special choice of the ground state, one can obtain the correct masses of gauge bosons.\\
Having introduced a new potential (the Higgs potential), the newly defined field (the Higgs field) must also have excited states, which can be interpreted as particles (the Higgs boson). In 2012, it has been announced by the ATLAS collaboration at CERN \parencite{Reference3}, that a neutral boson with a mass of 126 GeV with a significance of 5.9$\sigma$ has been found having the same properties as the Higgs boson. Thus, for this discovery, both Peter W. Higgs and François Englert have been awarded the Nobel prize the next year.\\
On the other hand however, some physical properties of this recently discovered particle are still unknown. In addition, several Beyond the Standard Model (BSM) theories predict the existence of multiple Higgs bosons (like the higgsino in case of SUSY), whose existence are still uncertain. One interesting property of the SM Higgs is its partity; as of 2018, excellent consistency has been found with the $J^{PC} = 0^{++}$ hypothesis \parencite{PDG_source} in the vector boson fusion (VBF) channel. Assuming a mixing angle between the CP-odd and CP-even states however, the precision of the parameters becomes fairly low \parencite{PDG}. Therefore, there is still ongoing research on the CP-even and CP-odd mixed states, which might help us -- in the long run -- to understand the matter-antimatter imbalance in the Universe.\\
In the following, this thesis is going to focus on the challenges and constraints of measuring the $CP$-parity of the Higgs boson by studying the impact parameter of the decay products in the gluon-gluon fusion channel.\\