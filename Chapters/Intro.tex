\chapter{Introduction} % Main chapter title

\label{Chapter1} % For referencing the chapter elsewhere, use \ref{Chapter1} 

The Standard Model (SM) of particle physics has proven to be a successful model to describe the elementary particles and their behaviour. Nevertheless, one had to wait long for the experimental verification of a theory predicting the masses of elementary particles (especially of the weak gauge bosons $W^\pm$ and $Z$) -- the Higgs mechanism.\\
Since the discovery of the Higgs boson in 2012 \parencite{Higgs2,Higgs1}, some of its properties (like the mass, spin and charge) of the newest SM particle have already been measured. Nevertheless, the parity quantum number of the Higgs boson is still an open question, as it has only been accessible in its vector boson couplings. For this purpose, the $H\rightarrow\tau\tau$ decay reactions which were observed during Run 2 of the CMS experiment at CERN at \SI{13}{\tera\electronvolt} energy scale of proton-proton collisions are going to be studied.\\
In these decay reactions, multiple pions and leptons can be produced in the final state. In cases where a $\tau$ decays into a pion or a muon, a decay plane can be constructed using the momentum of the child particle and its impact parameter with respect to the vertex of $\tau$ creation. It can be shown that the angle $\varphi_{CP}^*$ between the $\tau$ decay planes in the zero momentum frame is a CP-sensitive variable  \parencite{Berge_1prong, Berge_CP_Prospects}. Therefore, the correct determination of the impact parameter is of special interest.\\
The goal of this thesis is to find a quantity which helps to improve the event selection of the measurement. In addition, a new helical approach is going to be proposed which can help to improve the reconstruction of the impact parameter vectors.