% Chapter Help

\chapter{Conclusion and Outlook} % Main chapter title

\label{Chapter6} % For referencing the chapter elsewhere, use \ref{Chapter1} 
\label{sec:conclusion}

Since its discovery, the parity of the Higgs boson is still an open question. Even though provisory results are available for the vector boson coupling yielding a CP-even state, the question about the magnitude of the mixing angle (whose existence implies parity violation) in the $H\rightarrow\tau\tau$ decay channel described by the Yukawa-coupling is still open. One way of determing this property is the study of the decay planes of the $\tau$ decay products, which can be done in case of the one-prong and $\mu$ decay channels by studying the impact parameters of child particles.\\
In this thesis, a way to identify events with well-reconstructed decay planes and a new approach to determine the IP vector has been proposed. It has been found that events with an impact parameter vector lying outside the uncertainty ellipsoid of the primary vertex yield better reconstructed decay planes, and therefore, a cut based on this parameter delivers a better distribution with respect to the mixing angle $\varphi_\tau$.\\
In the second part of this work, an alternative way of IP vector reconstruction has been proposed. Although the new helical approach delivers similar results compared to the tangential approximation currently in use, a right, event-dependent combination of the these two should deliver the best results. A further study on their differences is therefore necessary in order to identify their capabilities.\\
There are two essential gains of both studies. First, both algorithms deliver a quantity with uncertainties which can be easily derived. Concerning the impact parameter magnitude, this would have meant in the case of the helical approach the error propagation of Eq. \ref{ansatz_sol_Claudia}. This now takes a more convenient form once one had regained the parameters of Eq. \ref{eq:ansatz} as a function of the five CMS-fit parameters. Secondly, a way has been found to determine the impact parameter vector from the reconstructed helical trajectory of the child particles.
In case of three-prong decays, where the secondary vertex can be reconstructed, the IP vector can be determined directly which can be done by applying a kinematical fit. Another way of improving the analysis would be the study of high-energy electron-positron collisions -- but this is a long term future perspective of this study.